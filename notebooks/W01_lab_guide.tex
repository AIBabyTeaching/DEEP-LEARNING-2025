\documentclass[11pt]{article}
\usepackage[margin=1in]{geometry}
\usepackage{hyperref}
\usepackage{enumitem}
\title{Lab 1 Guide: Deep Learning Foundations}
\author{Course Staff}
\date{\today}
\begin{document}
\maketitle

\section*{Objectives}
\begin{itemize}[leftmargin=*]
    \item Internalize how tensors, automatic differentiation, and mean squared error interlock to drive gradient-based learning~\cite{goodfellow2016deep,geron2022hands}.
    \item Translate analytic gradient derivations into NumPy and TensorFlow implementations that mirror the notebook demos.
    \item Practice reading optimizer diagnostics in TensorBoard to inform simple model design decisions.
\end{itemize}

\section*{Key Ideas}
\begin{itemize}[leftmargin=*]
    \item Tensors generalize scalars, vectors, and matrices so neural networks can manipulate structured data efficiently~\cite{goodfellow2016deep}.
    \item Automatic differentiation exposes gradients of composite computations without hand-derived calculus at every step~\cite{geron2022hands}.
    \item Mean squared error provides a smooth objective whose gradients connect the linear model warm-up to multilayer perceptrons.
\end{itemize}

\section*{Workflow}
\begin{enumerate}[leftmargin=*]
    \item Review the math recap linking MSE gradients to linear regression, then run the NumPy line-fit cell to see the closed-form solution in action.
    \item Step through the GradientTape mini-demo to observe how TensorFlow records operations and surfaces parameter updates automatically.
    \item Load Fashion-MNIST with the helper utilities, documenting each preprocessing stage (normalization, flattening, batching) as in the notebook comments.
    \item Train the baseline dense network, monitor early stopping and TensorBoard callbacks, and interpret the plotted history against your expectations from the math section.
\end{enumerate}

\section*{Assignments and Exit Ticket}
\begin{itemize}[leftmargin=*]
    \item Derive the gradients for $w$ and $b$ under MSE and compare with the notebook's analytical expressions.
    \item Run the hidden-size and optimizer comparisons, noting how the accuracy curves shift and how TensorBoard visualizes the differences.
    \item Exit Ticket: Submit a brief reflection connecting one GradientTape observation to the gradient narratives in Goodfellow et~al.~(2016) or Géron (2022).
\end{itemize}

\bibliographystyle{plain}
\begin{thebibliography}{9}
\bibitem{goodfellow2016deep}
Ian Goodfellow, Yoshua Bengio, and Aaron Courville.
\newblock \emph{Deep Learning}.
\newblock MIT Press, 2016.

\bibitem{geron2022hands}
Aur\'elien G\'eron.
\newblock \emph{Hands-On Machine Learning with Scikit-Learn, Keras, and TensorFlow}.
\newblock O'Reilly Media, 3rd edition, 2022.
\end{thebibliography}

\end{document}
